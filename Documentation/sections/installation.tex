\section{Telepítési lerírás}
\label{sec:installation}

\subsection{Backend telepítése}

\subsubsection{Fejlesztői környezet telepítése}

A backend fejlesztői környezet telepítéséhez szükséges a PHP és a Composer csomag kezelő telepítése. Az alkalmazás egyéb függőségei (adatbázis, stb) Docker kontérekekben futnak, így a Docker telepítése szükséges. A Docker telepítéséhez a \url{https://docs.docker.com/get-docker/} címről tölthető le a telepítő. A Composer telepítéséhez a \url{https://getcomposer.org/download/} címről tölthető le a telepítő. A Composer telepítése után a backend mappában a következő parancsot kell futtatni: \texttt{composer install}. Ezzel a Composer telepíti a Laravel függőségeket. 

A backend mappában a \texttt{.env.example} fájlt le kell másolni a \texttt{.env}-be. Ebben a fájlban kell beállítani az alkalmazás konfigurációját (adatbázis, frontend\_url, stb).

Ezután a \texttt{docker-compose.yaml} fájl segítségével indítható a backend. Ennek megkönnyítésére mi a Laravel Sail csomagot használtuk mely egy CLI a különböző konténerek kezelésére (lásd \cite{LaravelSail}). Indítás után szükséges az alkalmazás egyedi kulcsának generálása (\texttt{php artisan key:generate}, \texttt{sail} esetén a \texttt{php} kulcsszó minden további parancsban helyettesítendő \texttt{sail}-el), az adatbázis migrálása (\texttt{php artisan migrate}) és a jogosultságok inicializálása (\texttt{php artisan db:seed --class=PermissionSeeder}). Ezek után a backend elérhető a \url{http://localhost:8000} címen.

\subsubsection{Éles környezet telepítése}
Az éles környezetben való telepítéshez lásd a Laravel hivatalos dokumentációját: \url{https://laravel.com/docs/11.x/deployment}. Ezen kívűl az email küldés funkcióhoz szükséges egy SMTP szerver és annak konfigurációjának beállítása a \texttt{.env} fájlban.


\subsection{Android kliens telepítése}
A projekt továbbfejlesztéséhez vagy szerkesztéséhez Android stúdió telepítése szükséges. 
Az Android Studio letölthető a \url{https://developer.android.com/studio} címről. 
Az android studió tartalmazza az Android SDK-t ami a kód fordításához szükséges. A kód fordításához szükséges minimális SDK verziószáma: 24.
Az alkalmazás futtatásához szükséges egy Android eszköz vagy emulátor. 
Emulátor szintén telepíthető az Android studióban.

\par
Ha kiadásra szánjuk az alkalmazást, akkor le kell fordítani a brojektet és APK filet kell generálni. 
Az APK-t aláírással kell ellátni, majd feltölthető a Google Play áruházba. 
Innen a felhasználók letölthetik és telepíthetik az alkalmazást.

\subsection{Webes frontend telepítése}
A buildeléshez és fejlesztői módban való futtatáshoz npm szükséges.
A webes gyökérmappában (writer-reader-web-client) konzolon kiadva az \texttt{npm install} parancsot, beszerzi a szükséges JavaScript csomagokat.
Fejlesztői környezetben, a \texttt{npm run dev} parancsot kiadva elindíthatjuk.
A production verziót buildelni a \texttt{npm run build}-et használva tudjuk a writer-reader-web-client/dist alkönyvtárba.
Ezután hostolható egy szerveren. \url{https://vite.dev/guide/static-deploy.html}