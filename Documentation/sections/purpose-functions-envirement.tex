\section{A rendszer célja, funkciói, környezete.}

\subsection{Feladatkiírás}
A feladat egy író-olvasó oldal elkészítése. Az oldalon a regisztrált felhasználók olvashatják egymás megosztott történeteit, azokhoz megjegyzéseket, kritikákat fűzhetnek. A történeteket legyen lehetőség gyűjteményekbe, a fejezeteket regényekbe szervezni. A feltöltött történetek minden esetben moderátori ellenőrzésen esnek át, csak ez után érhetőek el publikusan. A moderálás eredményéről a felhasználót mindenképpen értesíteni kell. A történeteket el lehet látni jellemzőkkel, illetve meg lehet jelölni a kategóriájukat, a benne szereplő karaktereket, valamint figyelmeztetéseket és korhatárt lehet rájuk beállítani. Ezen kívül a regisztrált felhasználóknak lehetőségük van egymással privát üzenetben kommunikálni. A rendszerhez webes és mobilos kliens készítése is szükséges.
A részletes követelmények a \textit{Specifikáció} dokumentumban találhatók.
Mi a feladatkiírástól némileg eltérő nevezéktant használtunk: a továbbiekban a \textit{történet} helyett a \textit{mű} szót használjuk.

\subsection{A rendszer által biztosított funkciók}

A specifikáció alapján a platform a következő főbb funkciókat hivatott biztosítani. 
(Az egyes funkciók különböző szintű jogosultságokhoz kötöttek lehetnek.):
\begin{itemize}
    \item  Regisztráció
    \item  Bejelentkezés
    \item  Művekkel kapcsolatos funkciók:
    \begin{itemize}
        \item Történetek böngészése
        \item Történetek keresése
        \item Történetek olvasása
        \item Történetek létrehozása
        \item Történetek szerkesztése, fejezetekre osztása
        \item Történetek moderálása

    \end{itemize}
    \item  Gyűjteményekkel kapcsolatos funkciók:
    \begin{itemize}
        \item Gyűjtemények böngészése
        \item Gyűjtemények keresése
        \item Gyűjtemények megtekintése
        \item Gyűjtemények létrehozása 
    \end{itemize}
    \item  Hozzászólás írása Művekhez vgy Gyűjteményekhez
    \item  Művek, Gyűjtemények, Hozzászólások kedvelése
    \item  Privát üzenetek küldése
\end{itemize}

A rendszer az alábbi szerepköröket különbözteti meg: látogató, regisztrált felhasználó, moderátor, adminisztrátor. 
A látogatók csak a megosztott tartalmakat tekinthetik meg, a regisztrált felhasználók létrehozhatnak saját tartalmat, kommentelhetnek, kedvelhetnek és üzeneteket küldhetnek.
A moderátorok a moderálási jogosultságokkal rendelkeznek, az adminisztrátorok pedig a teljes rendszer felett rendelkeznek, kezelik a jogosultságokat.

\subsection{A rendszer környezete}

--- Backend és web környezetének leírása ---

A mobil kliens platformspecifikus, android operációs rendszerre készült kotlin nyelven Android Studioban.
A felhasználói felület normál méretű mobiltelefonra lett optimalizálva, egyéb méretű eszközökön (pl. okosórán) nem lett tesztelve.
A mobil kliens használatához internetelérés szükséges, az alkalmazás elérje  a backend szolgáltatásokat. 