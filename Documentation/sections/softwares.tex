\section{A program készítése során felhasznált eszközök}
\label{sec:softwares}

\subsection{Backend szoftverek}

\begin{itemize}
    \item Laravel 11, PHP 8.2, Composer
    \item Package-k: 
    \begin{itemize}
        \item FilamentPHP: admin felület,
        \item Laravel-Permissions: jogosultság kezelés,
        \item Laravel Breeze: authentikáció,
        \item Laravel Sail: Docker konténerek kezelése
        \item Pest: PHP tesztelés
    \end{itemize}
    \item Docker (Desktop): konténerizáció
    \item MySQL: adatbázis (konténerizálva)
    \item MailPit: mail küldés tesztelése (konténerizálva)
    \item PhpStorm IDE: fejlesztői környezet
\end{itemize}

\subsection{Webes szoftverek}

\begin{itemize}
    \item Vite 5.4 + React 18.3 (JSX)
    \item npm: JavaScript csomagkezelő
    \item Package-k: 
    \begin{itemize}
        \item React Router: navigáció
        \item MUI, Emotion, Roboto: UI React komponensek
        \item Axios: kérések
        \item js-cookie: sütik kezelése
        \item mui-markdown: markdown megjelenítése
        \item react-md-editor: markdown szerkesztő
        \item ESLint: kód analízis
    \end{itemize}
    \item Firefox, Chrome, MS Edge: böngészők
    \item Visual Studio Code: fejlesztői környezet
\end{itemize}