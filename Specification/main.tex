%--------------------------------------------------------------------------------------
% Dokumentum formátuma [Document format]
%--------------------------------------------------------------------------------------
\documentclass[12pt,a4paper,oneside]{article}

%--------------------------------------------------------------------------------------
% Csomagok inicializálása [Initializing packages]
%--------------------------------------------------------------------------------------

\usepackage[utf8]{inputenc}
\usepackage[magyar]{babel}
\usepackage{t1enc}
\usepackage{amsmath}
\usepackage{amsfonts}
\usepackage{amssymb}
\usepackage{geometry}
\geometry{
    top=20mm,
    bottom=20mm,
    left=20mm,
    right=20mm,
}
\usepackage{hyperref}
\usepackage{mathrsfs}
\usepackage{fancyhdr}
\usepackage{graphicx}
\usepackage{icomma}
\usepackage{adjustbox}
\usepackage{lastpage}
\usepackage{float}
\usepackage{pdfpages}
\usepackage{subcaption}
\usepackage{tocloft}
\usepackage{bm}
\usepackage{multicol}
\usepackage{listings}
\usepackage{tikz}
\usepackage[abs]{overpic}
\usepackage{pict2e}
\usepackage{wrapfig} 

%--------------------------------------------------------------------------------------
% Dokumentum törzse [Document body]
%--------------------------------------------------------------------------------------

\begin{document}

%--------------------------------------------------------------------------------------
% Címoldal [Titlepage]
%--------------------------------------------------------------------------------------

\author{\textit{Készítették:}\\ Buczny Dominik \hspace{5pt} Hanich Péter \\ Németh Gergő Olivér \hspace{5pt} Olchváry Ambrus \\[10pt] \textit{Konzulens:}\\ Gazdi László}
\title{Szoftverarchitektúrák Házi Feladat \\ Közösségi író-olvasó oldal \\ \textbf{Dokumentáció}}
\date{2024/2025/1}
\maketitle
\thispagestyle{empty}
\begin{center}
    \includegraphics[width=\textwidth,height=\textheight,keepaspectratio]{./figures/logo.png}
\end{center}  

%--------------------------------------------------------------------------------------
% Tartalomjegyzék [Table of Contents]
%--------------------------------------------------------------------------------------
\newpage
\tableofcontents
\thispagestyle{empty}
\newpage

%--------------------------------------------------------------------------------------
% Oldalszámok formázása [Page numbering]
%--------------------------------------------------------------------------------------
\fancyhf{}
\lhead{Szoftverarchitektúrák}
\rhead{VIAUMA21}
\cfoot{\thepage}
\pagestyle{fancy}
\setcounter{page}{1}


%--------------------------------------------------------------------------------------
% Tartalom [Contents]
%--------------------------------------------------------------------------------------

%TODO A tartalmat ide írd  [Write the chapters here]
% Ki lehet file-okba is szervezni [You can organize it into files]
% Példa: \include{sections/introduction}
\section{Fogalomszótár}
\begin{description}
\item [Látogató:] Egy személy, aki az író olvasó klienst használja. Képes a megosztott művek között böngészni és azokat olvasni.

\item [(Regisztrált) Felhasználó:] Egy látogató, aki regisztrál a platformon, és képes műveket olvasni, megosztani, kommentelni, saját műveit gyűjteményekbe szervezni, valamint privát üzeneteket küldeni.

\item [Moderátor:] Egy speciális felhasználó, aki jogosult a beküldött történetek átnézésére, jóváhagyására vagy elutasítására.

\item [Admin:] Minden jogosultsággal rendelkező speciális felhasználó, moderátorokat jelölhet ki, a felhasználókat menedzselheti

\item [Mű:] A platformon megosztott írásos tartalom. A tartalma fejezeteket és fejezeten kívüli szöveges részeket tartalmazhat. (pl: a regény fejezetekből áll, egy vers vagy novella nem tartalmaz fejezeteket).
A művek gyűjtemények részei lehetnek. Művekhez kommentek fűzhetők és kedvelhetők
A műveket saját szerzőik és a moderátorok opcionálisan tulajdonságokkal láthatják el:

\begin{itemize}
\item \emph{Kategória:} Leírja a mű műfaját vagy típusát.
Előre definiált kategóriák vannak amiket a felhasználók használhatnak
Kategóriát moderátor hozhat létre, módosíthat, törölhet
\item \emph{Címkék:} Kulcsszavak, amik további információt adnak a műről.
Bárki hozhat létre új jellemzőt, vagy használhat egy már meglévőt
\item \emph{Szereplők:} A műben szereplő kitalált vagy valós személyek.
Bárki hozhat létre új szereplőt, vagy használhat egy már meglévőt 
\item \emph{Figyelmeztetések:} Specifikus tartalmi figyelmeztetések (pl: erőszakos tartalom).
Előre definiált figyelmeztetések vannak amiket a felhasználók használhatnak
Figyelmeztetést moderátor hozhat létre, módosíthat, törölhet
\item \emph{Korhatár:} Jelzi a mű olvasására ajánlott korosztályt.
Előre definiált korhatárok vannak amiket a felhasználók használhatnak
Korhatárt moderátor hozhat létre, módosíthat, törölhet
\item \emph{Nyelv:} Az a nyelv, amelyen a mű olvasható.
Előre felvett listából választható, a földön beszélt legtöbb nyelvet tartalmazza
\end{itemize}

\item [Fejezet:]  Művek alegységei, melyek saját címmel és számozással rendelkeznek. Egy fejezet nem bontható további alegységekre. A műveket nem kötelező fejezetekre bontani. 

\item [Gyűjtemény:] Publikált művek egy listája, címmel ellátva. A felhasználók a publikált műveket gyűjteményekbe rendezhetik. A gyűjtemények nyilvánosan elérhetők, de csak az azt létrehozó felhasználó szerkesztheti.

\item [Megjegyzések:] A történetekhez, gyűjteményekhez, regényekhez hozzáfűzhető vélemények és kritikák. Kommentekhez nem lehet válasz kommentet fűzni, de lehet kedvelni.

\item [Kedvelések:] Egy felhasználó műveket és megjegyzéseket is kedvelhet, de egy adottat csakis egyszer. Több fajta reakció (pl: dislike, emoji) nincs.

\item [Privát üzenet:] Egy kommunikációs funkció ami lehetővé teszi a felhasználóknak, hogy személyes üzeneteket küldjenek egymásnak.
\end{description}

\section{Kliens funkciók}

\subsection{Szereplők:}
\begin{itemize}
\item Látogató
\item (Regisztrált) Felhasználó
\item Moderátor
\item Rendszer
\end{itemize}

\subsection{Use case-ek:}
\textbf{(w):} csak webes, \textbf{(m):} csak mobilos funkció
\begin{itemize}
\item \textbf{Regisztráció/Bejelentkezés:} A klienst használó személy létrehoz egy felhasználói fiókot vagy bejelentkezik.
\item \textbf{Felhasználói fiók beállítások:} A felhasználó módosíthatja profiljának adatait (pl. felhasználónév, jelszó)
\item \textbf{Történet létrehozása / szerkesztése / beküldése moderálásra (w):} Felhasználó új történeteket hozhat létre, korábbi történeteit szerkesztheti, beküldheti történeteit moderálásra.
\item \textbf{Történetek olvasása:} Látogatók böngészhetnek a közzétett történetek között és megnyithatják őket olvasásra.
\item \textbf{Történethez hozzászólás/kritika rendelése:} A felhasználó bármely publikált műhöz megjegyzést írhat
\item \textbf{Történet letöltése offline olvasásra (m):} Felhasználók letölthetnek történeteket amik törlésig lokálisan a mobil eszközön tárolódnak és megnyithatók olvasásra.  
\item \textbf{Letöltött történetek törlése (m):} Felhasználó törölheti az eszközére korábban letöltött történeteket
\item \textbf{Moderálás:} Az a folyamat, amely során a beküldött történeteket a moderátorok átnézik, jóváhagyják vagy elutasítják. 
\item \textbf{Moderálási értesítés fogadása:} A moderálás eredményéről a felhasználó számára érkező automata üzenet.
\item \textbf{Gyűjteménybe/fejezetbe rendezés:} Egy felhasználó saját gyűjteményt hozhat létre, melybe saját és mások publikusan elérhető műveit helyezheti el. Egy felhasználó csak a saját gyűjteményeit szerkesztheti. 
\item \textbf{Privát üzenetek küldése fogadása:} Felhasználók egymásnak üzeneteket küldhetnek
\end{itemize}
\subsection{Use case diagram:}
\begin{center}
    \includegraphics[width=\textwidth,height=\textheight,keepaspectratio]{./figures/Use-case_diagram.png}
\end{center}


\section{Architektúra}
Teszt szöveg.\cite{wikipedia}

%--------------------------------------------------------------------------------------
% Bibliográfia [Bibliography] - Ha üres kommenteld ki [Uncomment if empty]
%--------------------------------------------------------------------------------------
\newpage
\thispagestyle{empty}
\pagenumbering{gobble}
{
    \footnotesize  % Kisebb betűméret [Smaller font size]
    \bibliographystyle{plain}
    \bibliography{mybib}
}
\addcontentsline{toc}{section}{Hivatkozások}

\end{document}