\section{Fogalomszótár}
\textbf{Látogató:} Egy személy, aki az író olvasó klienst használja. Képes a megosztott művek között böngészni és azokat olvasni. 

\textbf{(Regisztrált) Felhasználó:} Egy látogató, aki regisztrál a platformon, és képes műveket olvasni, megosztani, kommentelni, saját műveit gyűjteményekbe szervezni, valamint privát üzeneteket küldeni.

\textbf{Moderátor:} Egy speciális felhasználó, aki jogosult a beküldött történetek átnézésére, jóváhagyására vagy elutasítására.

\textbf{Admin:} Minden jogosultsággal rendelkező speciális felhasználó, moderátorokat jelölhet ki, a felhasználókat menedzselheti

\textbf{Mű:} A platformon megosztott írásos tartalom. A tartalma fejezeteket és fejezeten kívüli szöveges részeket tartalmazhat. (pl: a regény fejezetekből áll, egy vers vagy novella nem tartalmaz fejezeteket). A művek gyűjtemények részei lehetnek. Művekhez kommentek fűzhetők és kedvelhetők. A műveket saját szerzőik és a moderátorok opcionálisan tulajdonságokkal láthatják el:
\begin{itemize}
    \item \textbf{Kategória:} Leírja a mű műfaját vagy típusát.
    \begin{itemize}
        \item Előre definiált kategóriák vannak amiket a felhasználók használhatnak
        \item Kategóriát moderátor hozhat létre, módosíthat, törölhet
    \end{itemize}
    \item \textbf{Címkék:} Kulcsszavak, amik további információt adnak a műről.
    \begin{itemize}
        \item Bárki hozhat létre új jellemzőt, vagy használhat egy már meglévőt
    \end{itemize}
    \item \textbf{Szereplők:} A műben szereplő kitalált vagy valós személyek.
    \begin{itemize}
        \item Bárki hozhat létre új szereplőt, vagy használhat egy már meglévőt 
    \end{itemize}
    \item \textbf{Figyelmeztetések:} Specifikus tartalmi figyelmeztetések (pl: erőszakos tartalom).
    \begin{itemize}
        \item Előre definiált figyelmeztetések vannak amiket a felhasználók használhatnak
        \item Figyelmeztetést moderátor hozhat létre, módosíthat, törölhet
    \end{itemize}
    \item \textbf{Korhatár:} Jelzi a mű olvasására ajánlott korosztályt.
    \begin{itemize}
        \item Előre definiált korhatárok vannak amiket a felhasználók használhatnak
    \end{itemize}
    \item \textbf{Nyelv:} Az a nyelv, amelyen a mű olvasható.
    \begin{itemize}
        \item Előre felvett listából választható, a földön beszélt legtöbb nyelvet tartalmazza
    \end{itemize}
\end{itemize}

\textbf{Fejezet:}  Művek alegységei, melyek saját címmel és számozással rendelkeznek. Egy fejezet nem bontható további alegységekre. A műveket nem kötelező fejezetekre bontani. 

\textbf{Gyűjtemény:} Publikált művek egy listája, címmel ellátva. A felhasználók a publikált műveket gyűjteményekbe rendezhetik. A gyűjtemények nyilvánosan elérhetők, de csak az azt létrehozó felhasználó szerkesztheti.

\textbf{Megjegyzések:} A történetekhez, gyűjteményekhez, regényekhez hozzáfűzhető vélemények és kritikák. Kommentekhez nem lehet válasz kommentet fűzni, de lehet kedvelni.

\textbf{Kedvelések:} Egy felhasználó műveket és megjegyzéseket is kedvelhet, de egy adottat csakis egyszer. Több fajta reakció (pl: dislike, emoji) nincs.

\textbf{Privát üzenet:} Egy kommunikációs funkció ami lehetővé teszi a felhasználóknak, hogy személyes üzeneteket küldjenek egymásnak.