\section{Fogalomszótár}

\begin{description}
\item [Látogató:] Egy személy, aki az író olvasó klienst használja. Képes a megosztott művek között böngészni és azokat olvasni.

\item [(Regisztrált) Felhasználó:] Egy látogató, aki regisztrál a platformon, és képes műveket olvasni, megosztani, kommentelni, saját műveit gyűjteményekbe szervezni, valamint privát üzeneteket küldeni.

\item [Moderátor:] Egy speciális felhasználó, aki jogosult a beküldött történetek átnézésére, jóváhagyására vagy elutasítására.

\item [Admin:] Minden jogosultsággal rendelkező speciális felhasználó, moderátorokat jelölhet ki, a felhasználókat menedzselheti

\item [Mű:] A platformon megosztott írásos tartalom. A tartalma fejezeteket és fejezeten kívüli szöveges részeket tartalmazhat. (pl: a regény fejezetekből áll, egy vers vagy novella nem tartalmaz fejezeteket).
A művek gyűjtemények részei lehetnek. Művekhez kommentek fűzhetők és kedvelhetők
A műveket saját szerzőik és a moderátorok opcionálisan tulajdonságokkal láthatják el:

\begin{itemize}
\item \emph{Kategória:} Leírja a mű műfaját vagy típusát.
Előre definiált kategóriák vannak amiket a felhasználók használhatnak.
Kategóriát moderátor hozhat létre, módosíthat, törölhet
\item \emph{Címkék:} Kulcsszavak, amik további információt adnak a műről.
Bárki hozhat létre új jellemzőt, vagy használhat egy már meglévőt
\item \emph{Szereplők:} A műben szereplő kitalált vagy valós személyek.
Bárki hozhat létre új szereplőt, vagy használhat egy már meglévőt 
\item \emph{Figyelmeztetések:} Specifikus tartalmi figyelmeztetések (pl: erőszakos tartalom).
Előre definiált figyelmeztetések vannak amiket a felhasználók használhatnak
Figyelmeztetést moderátor hozhat létre, módosíthat, törölhet
\item \emph{Korhatár:} Jelzi a mű olvasására ajánlott korosztályt.
Előre definiált korhatárok vannak amiket a felhasználók használhatnak
Korhatárt moderátor hozhat létre, módosíthat, törölhet
\item \emph{Nyelv:} Az a nyelv, amelyen a mű olvasható.
Előre felvett listából választható, a földön beszélt legtöbb nyelvet tartalmazza
\end{itemize}

\item [Fejezet:]  Művek alegységei, melyek saját címmel és számozással rendelkeznek. Egy fejezet nem bontható további alegységekre. A műveket nem kötelező fejezetekre bontani. 

\item [Gyűjtemény:] Publikált művek egy listája, címmel ellátva. A felhasználók a publikált műveket gyűjteményekbe rendezhetik. A gyűjtemények nyilvánosan elérhetők, de csak az azt létrehozó felhasználó szerkesztheti.

\item [Megjegyzések:] A történetekhez, gyűjteményekhez, regényekhez hozzáfűzhető vélemények és kritikák. Kommentekhez nem lehet válasz kommentet fűzni, de lehet kedvelni.

\item [Kedvelések:] Egy felhasználó műveket és megjegyzéseket is kedvelhet, de egy adottat csakis egyszer. Több fajta reakció (pl: dislike, emoji) nincs.

\item [Privát üzenet:] Egy kommunikációs funkció ami lehetővé teszi a felhasználóknak, hogy személyes üzeneteket küldjenek egymásnak.
\end{description}