\section{Kliens funkciók}

\subsection{Szereplők:}
\begin{itemize}
\item Látogató
\item (Regisztrált) Felhasználó
\item Moderátor
\item Rendszer
\end{itemize}

\subsection{Use case-ek:}
\textbf{(w):} csak webes, \textbf{(m):} csak mobilos funkció
\begin{itemize}
\item \textbf{Regisztráció/Bejelentkezés:} A klienst használó személy létrehoz egy felhasználói fiókot vagy bejelentkezik.
\item \textbf{Felhasználói fiók beállítások:} A felhasználó módosíthatja profiljának adatait (pl. felhasználónév, jelszó)
\item \textbf{Történet létrehozása / szerkesztése / beküldése moderálásra (w):} Felhasználó új történeteket hozhat létre, korábbi történeteit szerkesztheti, beküldheti történeteit moderálásra.
\item \textbf{Történetek olvasása:} Látogatók böngészhetnek a közzétett történetek között és megnyithatják őket olvasásra.
\item \textbf{Történethez hozzászólás/kritika rendelése:} A felhasználó bármely publikált műhöz megjegyzést írhat
\item \textbf{Történet letöltése offline olvasásra (m):} Felhasználók letölthetnek történeteket amik törlésig lokálisan a mobil eszközön tárolódnak és megnyithatók olvasásra.  
\item \textbf{Letöltött történetek törlése (m):} Felhasználó törölheti az eszközére korábban letöltött történeteket
\item \textbf{Moderálás:} Az a folyamat, amely során a beküldött történeteket a moderátorok átnézik, jóváhagyják vagy elutasítják. 
\item \textbf{Moderálási értesítés fogadása:} A moderálás eredményéről a felhasználó számára érkező automata üzenet.
\item \textbf{Gyűjteménybe/fejezetbe rendezés:} Egy felhasználó saját gyűjteményt hozhat létre, melybe saját és mások publikusan elérhető műveit helyezheti el. Egy felhasználó csak a saját gyűjteményeit szerkesztheti. 
\item \textbf{Privát üzenetek küldése fogadása:} Felhasználók egymásnak üzeneteket küldhetnek
\end{itemize}
\subsection{Use case diagram:}
\begin{center}
    \includegraphics[width=\textwidth,height=\textheight,keepaspectratio]{./figures/Use-case_diagram.png}
\end{center}