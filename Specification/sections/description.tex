\section{Feladatleírás}

\subsection{Feladatkiírás}

A feladat egy író-olvasó oldal elkészítése. Az oldalon a regisztrált felhasználók olvashatják egymás megosztott történeteit, azokhoz megjegyzéseket, kritikákat fűzhetnek. A történeteket legyen lehetőség gyűjteményekbe, a fejezeteket regényekbe szervezni. A feltöltött történetek minden esetben moderátori ellenőrzésen esnek át, csak ez után érhetőek el publikusan. A moderálás eredményéről a felhasználót mindenképpen értesíteni kell. A történeteket el lehet látni jellemzőkkel, illetve meg lehet jelölni a kategóriájukat, a benne szereplő karaktereket, valamint figyelmeztetéseket és korhatárt lehet rájuk beállítani. Ezen kívül a regisztrált felhasználóknak lehetőségük van egymással privát üzenetben kommunikálni. A rendszerhez webes és mobilos kliens készítése is szükséges.

\subsection{A fejlesztői csapat}
A csapat tagjai:
\begin{table}[h]
    \begin{tabular}{ |l|l|l|l| }
        \hline
        \textbf{Csapattag neve} & \textbf{Neptun-kódja} & \textbf{E-mail címe}
        & \textbf{Általános feladatköre} \\
        \hline\hline
        Buczny Dominik & C84AVJ & \href{mailto:bucznydominik@gmail.com}{bucznydominik@gmail.com}
        & webes frontend \\
        \hline
        Hanich Péter & A1FVB8 & \href{mailto:hanipet@gmail.com}{hanipet@gmail.com}
        & backend és adatbázis \\ %TODO
        \hline
        Németh Gergő Olivér & PIBK75 & \href{mailto:nemeth.n.gergo@gmail.com}{nemeth.n.gergo@gmail.com}
        & mobil kliens \\ %TODO
        \hline
        Olchváry Ambrus & AZLFR2 & \href{mailto:ambrus.olchvary@gmail.com}{ambrus.olchvary@gmail.com} & backend és adatbázis \\ %TODO
        \hline
    \end{tabular}
\end{table}

\subsection{Részletes feladatleírás}

%TODO

\subsection{Technikai paraméterek}

Az alkalmazás egy webes és egy mobilos kliensből áll, valamint egy backendből és az általa használt adatbázisból. A webes kliens  React keretrendszerre épül, a mobilos kliens pedig natív Androidra készül Jetpack Compose keretrendszerrel írt felhasználói felülettel. A backend a Laravel keretrendszerrel készül, az adatbázis pedig valamilyen SQL alapú relációs adatbázis lesz. A könyebb telepítés és futtatás érdekében Docker konténereket használunk.